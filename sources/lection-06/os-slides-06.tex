\documentclass[xcolor=table]{beamer}
\mode<presentation>
\usetheme{CambridgeUS}
\usepackage[russian]{babel}
\usepackage[utf8]{inputenc}
\usepackage[T2A]{fontenc}
\usepackage{sansmathaccent}
\usepackage{verbatim}
\usepackage{alltt}
\usepackage[table]{xcolor}
%\linespread{0.8}
\usepackage{minted}
\usepackage{setspace}

\pdfmapfile{+sansmathaccent.map}
\title[Язык C]{Низкоуровневый ввод-вывод, часть 5}
\author{Наумов Д.А., доц. каф. КТ}
\date[14.10.2019] {Операционные системы и системное программное обеспечение, 2020}

\begin{document}

%ТИТУЛЬНЫЙ СЛАЙД
\begin{frame}
  \titlepage
\end{frame}
  
%СОДЕРЖАНИЕ ЛЕКЦИИ
\begin{frame}
  \frametitle{Содержание лекции}
  \tableofcontents  
\end{frame}

\section{Файловый ввод-вывод}

\subsection{Открытие файлов}

\begin{frame}{Файловый ввод-вывод}
\begin{itemize}
\item Ядро поддерживает попроцессный список открытых файлов, называемый \textbf{файловой таблицей}. 
\item Таблицы индексируется с помощью неотрицательных целых чисел, называемых \textbf{файловыми дескрипторами} (часто они именуются сокращенно \textbf{fd}). 
\item Каждая запись в списке содержит информацию об открытом файле, в частности указатель на хранимую в памяти копию файлового дескриптора и ассоциированных с ним метаданных.
\item К метаданным относятся:
	\begin{itemize}
	\item файловая позиция;
	\item режимы доступа.
	\end{itemize}
\end{itemize}
\end{frame}

\begin{frame}{Стандартные дескрипторы}
\begin{itemize}
\item Каждый процесс традиционно имеет не менее трех открытых файловых дескрипторов: 0, 1 и 2. 
	\begin{itemize}
	\item дескриптор \textbf{0} соответствует стандартному вводу (\textbf{stdin}), 
	\item дескриптор \textbf{1} соответствует стандартному выводу (\textbf{stdout}), 
	\item дескриптор \textbf{2} соответствует стандартной ошибке (\textbf{stderr}).
	\end{itemize}

\item Библиотека С не ссылается непосредственно на эти целые числа, а предоставляет препроцессорные определения:
	\begin{itemize}
	\item STDIN\_FILENO 
	\item STDOUT\_FILENO 
	\item STDERR\_FILENO
	\end{itemize} 
\item Как правило, \textbf{stdin} подключен к терминальному устройству ввода (обычно это пользовательская клавиатура), а \textbf{stdout} и \textbf{stderr} — к дисплею терминала. 
\end{itemize}
\end{frame}

\begin{frame}[fragile]{Системный вызов open()}
	Открытие файла и получение файлового дескриптора осуществляются с помощью системного вызова open():
	\begin{minted}{c}
	#include <sys/types.h>
	#include <sys/stat.h>        
	#include <fcntl.h>                

	int open (const char *name, int flags);
	int open (const char *name, int flags, mode_t mode);		
	\end{minted}  
	\begin{itemize}
	\item Системный вызов \textbf{open()} ассоциирует файл, на который указывает имя пути \textbf{name} с файловым дескриптором, возвращаемым в случае успеха.
	\item В качестве файловой  позиции указывается его начало (нуль), и файл открывается для доступа в соответствии с заданными флагами (параметр \textbf{flags}).
	\end{itemize}	  
\end{frame}

\begin{frame}[fragile]{Флаги для открытия файла}
	\begin{itemize}
		\item Аргумент \textbf{flags} -- это поразрядное ИЛИ, состоящее из одного или нескольких флагов, определяющее режим доступа к файлу. 
		\item Режим доступа может иметь одно из следующих значений: O\_RDONLY, O\_WRONLY или O\_RDWR.
	\end{itemize}
	\begin{minted}{c}
	int fd;

	fd = open("/home/kidd/madagascar", O_RDONLY);
	if (fd == -1)
		/*ошибка*/
	\end{minted}  
	\begin{itemize}
		\item Если файл открыт только для чтения, в него невозможно что­-либо записать, и наоборот. 
		\item Процесс, осуществляющий системный вызов \textbf{open()}, должен иметь права, чтобы получить запрашиваемый доступ. 
	\end{itemize}
\end{frame}

\begin{frame}{Открытие файла}
	Некоторые дополнительные флаги аргумента \textbf{flags} для изменения поведения \textbf{open()}. 
	\begin{itemize}
		\item \textbf{O\_APPEND} -- режим дозаписи.
		
		\medskip Перед каждым актом записи файловая позиция будет обновляться и устанавливаться в текущий конец файла.
		\item \textbf{O\_CREAT}. Если файл, обозначаемый именем \textbf{name}, не существует, то ядро создаст его. 
		\item \textbf{O\_TRUNC}. Если файл уже существует, является обычным файлом и заданные для него флаги допускают запись, то файл будет усечен до нулевой длины. 
	\end{itemize}
\end{frame}

\begin{frame}{Открытие файла}
	Дополнительные флаги аргумента \textbf{flags} для изменения поведения \textbf{open()}. 
	\begin{itemize}
	\item \textbf{O\_ASYNC}
	
	\medskip Когда указанный файл станет доступным для чтения или записи, генерируется специальный сигнал (по умолчанию \textbf{SIGIO}). Этот флаг может 	использоваться только при работе с FIFO, каналами, сокетами и терминалами, но не с обычными файлами.
	\item \textbf{O\_SYNC}
	
	\medskip Файл будет открыт для синхронного ввода­вывода. Никакие операции записи не завершатся, пока данные физически не окажутся на диске. 
\end{itemize}
\end{frame}

\begin{frame}[fragile]{Права доступа создаваемых файлов}
	Аргумент \textbf{mode} требуется, если задан флаг \textbf{O\_CREAT}. 
	\begin{minted}{c}
  int fd;

  fd = open(file, O_WRONLY | O_CREAT | O_TRUNC,
            S_IWUSR | S_IRUSR | S_IWGRP | S_IRGRP | S_IROTH);
  if (fd == -1)
     /*ошибка*/
	\end{minted}  
	\begin{itemize}
		\item при создании файла аргумент \textbf{mode} задает права доступа к этому новому файлу;
		\item режим доступа не проверяется при данном конкретном открытии файла;
		\item аргумент \textbf{mode} является UNIX-­последовательностью битов, регламентирующей доступ.
	\end{itemize}
\end{frame}

\begin{frame}[fragile]{Функция creat()}
	Комбинация O\_WRONLY | O\_CREAT | O\_TRUNC настолько распространена, что существует специальный системный вызов, обеспечивающий именно такое поведение: 
  \begin{minted}{c}
  #include <sys/types.h>
  #include <sys/stat.h>        
  #include <fcntl.h>     
  int creat (const char *name, mode_t mode);
  \end{minted}  
	В большинстве архитектур \textit{Linux} \textbf{creat()} является системным вызовом, хотя его можно легко реализовать и в пользовательском пространстве:

\begin{minted}{c}
  int creat (const char *name, mode_t mode)
  {
    return open(name, O_WRONLY | O_CREAT | O_TRUNC, mode);
  }
\end{minted}

	При ошибке \textit{open}() и \textit{creat}() возвращают \textbf{–1} и устанавливают \textit{errno}.
\end{frame}

\subsection{Чтение данных}

\begin{frame}[fragile]{Системный вызов read()}
\begin{minted}{c}
  #include <unistd.h>
  ssize_t read (int fd, void *buf, size_t len);
 \end{minted}
 
	\linespread{0.9}
	\begin{itemize}
		\item Каждый вызов считывает не более \textbf{len} байт в памяти, на которые содержится	указание в \textbf{buf}. 
		\item Считывание происходит с текущим значением смещения, в файле, указанном в \textbf{fd}. 
		\item При успешном вызове возвращается количество байтов, записанных в \textbf{buf}. 
		\item При ошибке вызов возвращает \textbf{-1} и устанавливает \textbf{errno}. 
		\item Файловая позиция продвигается в зависимости от того, сколько байтов было считано с \textbf{fd}. 
		\item Если объект, указанный в \textbf{fd}, не имеет возможности позиционирования (например, это файл символьного устройства), то считывание всегда начинается с <<текущей>> позиции.
	\end{itemize}
	\linespread{1.0}
\end{frame}

\begin{frame}[fragile]{Системный вызов read()}
\begin{minted}{c}
  unsigned long word;
  ssize_t nr;
  
  /*считываем пару байт в 'word' из 'fd'*/
  nr = read(fd, &word, sizeof(unsigned long));
  if (nr == -1)
      /*ошибка*/
\end{minted}

\begin{itemize}
\item Вызов может вернуться, считав не все байты из \textit{len}; 
\item Могут возникнуть ошибки, требующие исправления, но не проверяемые и не обрабатываемые в коде.
\end{itemize}
\end{frame}

\begin{frame}{Возможные последствия вызова read()}
	Вызов взвращает значение:
	\begin{itemize}
		\item равное \textbf{len} -- все \textit{len} считанных байтов сохраняются в \textit{buf}. 
		\item меньшее, чем len, но большее чем нуль -- 
			\begin{enumerate}
				\item cчитанные байты сохраняются в buf;
				\item ошибка возникает в середине процесса; 
				\item возвращается значение, большее 0, но меньшее len. 
				\item например: конец файла был достигнут ранее, чем было прочитано заданное количество байтов. 
				\item при повторном вызове (в котором соответствующим образом обновлены значения len и buf) оставшиеся байты будут считаны в оставшуюся часть буфера, либо укажут на причину проблемы.
			\end{enumerate}		 
		\item равное 0 -- достигнут конец файла, считывать больше нечего.
	\end{itemize}
\end{frame}

\begin{frame}{Возможные последствия вызова read()}
	Вызов взвращает значение:
	\begin{itemize}
	\item равное \textbf{–1}, а \textbf{errno = EINTR} -- сигнал был получен прежде, чем были считаны какие­-либо байты. Вызов будет повторен.
	\item равное \textbf{–1}, а \textbf{errno = EAGAIN} -- вызов блокировался потому, что в настоящий момент нет доступных данных, и запрос следует повторить позже (это происходит только в неблокирующем режиме).
	\item равное \textbf{–1}, а \textbf{errno != EINTR, errno != EAGAIN} -- более серьезная ошибка (простое повторение вызова в данном случае, скорее всего, не поможет).
	\end{itemize}
\end{frame}

\begin{frame}[fragile]{Считывание всех байт}
\begin{minted}{c}
  ssize_t ret;
  while (len != 0 && (ret = read(fd, buf, len)) !=0 )  
  {
    if (ret == -1)
    {
      if (errorno == EINTR)
        continue;
      perror("read");
      break;
    }
    len -= ret;
    buf += ret;
  }
\end{minted}
\end{frame}

\begin{frame}[fragile]
\textbf{Неблокирующий ввод-вывод}: вызов read() не блокируется при отсутствии доступных данных, вместо этого -- немедленный возврат вызова, указывающий, что данных действительно нет. 

\begin{minted}{c}
  char buf[BUFSIZE];
  ssize_t nr;
  
  start:
  nr = read(fd, buf, BUFSIZE);
  
  if (nr == -1){
      if (errorno == EINTR)
        goto start; /*пытемся считать еще раз*/
      if (errorno == EAGAIN)
        /*повторить вызов позже*/
      else
        /*ошибка*/
  }
\end{minted}
\end{frame}

\subsection{Запись данных}

\begin{frame}[fragile]{Системный вызов write()}
	\begin{minted}{c}
	#include <unistd.h>
	
	ssize_t write (int fd, const void *buf, size_t len);
	\end{minted}
	
	\begin{itemize}
		\item при вызове \textbf{write()} записывается некоторое количество байтов, меньшее или равное тому, что указано в \textbf{count}. 
		\item запись начинается с \textbf{buf}, установленного в текущую файловую позицию. 
		\item при успешном выполнении возвращается количество записанных байтов, а файловая позиция обновляется соответственно. 
		\item при ошибке возвращается \textbf{-1} и устанавливается соответствующее значение \textbf{errno}.
	\end{itemize}
\end{frame}

\begin{frame}[fragile]{Простейшиый пример использования}
	\begin{minted}{c}
	const char *buf = "My ship is solid!";
	
	/*строка, находящаяся в buf, записывается в fd*/
	nr = write(fd, buf, strlen(buf));
	
	if (nr == -1)
	  /*ошибка*/
	\end{minted}
\end{frame}

\begin{frame}[fragile]{Проверка всех ошибок}
	\begin{minted}{c}
	unsigned long word = 1720;
	
	size_t count;
	ssize_t nr;
	
	count = sizeof(word);
	nr = write(fd, &word, count);
	
	if (nr == -1)
	  /* ошибка, проверить errorno */
	else if (nr != count)
	  /* возмжна ошибка, 
	     но значение errorno не установлено */
	\end{minted}
\end{frame}

\begin{frame}{Режимы записи и ошибки}
	Режим дозаписи: 
	\begin{itemize}
		\item Когда дескриптор \textbf{fd} открывается в режиме дозаписи (с флагом O\_APPEND), запись начинается не с \textit{текущей} позиции дескриптора файла, а с точки, в которой в данный момент находится конец файла.
	\end{itemize}

	\medskip
	Неблокирующая запись: 
	\begin{itemize}
		\item Когда дескриптор \textbf{fd} открывается в неблокирующем режиме (с флагом O\_NONBLOCK), а запись в том виде, в котором она выполнена, в  нормальных условиях должна быть заблокирована, системный вызов \textbf{write()} возвращает \textbf{–1} и устанавливает \textbf{errno} в значение \textbf{EAGAIN}. Запрос следует повторить позже.
	\end{itemize}	
\end{frame}

\section{Синхронный ввод-вывод}
\begin{frame}
  \frametitle{Содержание лекции}
  \tableofcontents[current]
\end{frame}

\begin{frame}[fragile]{Синхронный ввод-вывод}
	\begin{block}{fsync()}
		метод, позволяющий гарантировать, что данные окажутся на диске -- использовать системный вызов fsync().
	\end{block}		
		
	\begin{minted}{c}
	#include <unistd.h>
	
	int fsync(int fd);
	\end{minted}
	
	\begin{itemize}
		\item Вызов \textbf{fsync}() гарантирует, что все <<грязные>> данные, ассоциированные с конкретным файлом, на который отображается дескриптор \textbf{fd}, 	будут записаны на диск.
		\item Файловый дескриптор \textbf{fd} должен быть открыт для записи. 
		\item Вызов заносит на диск как данные, так и метаданные (цифровые отметки о времени создания файла и другие атрибуты).
		\item Вызов \textbf{fsync()} не вернется, пока жесткий диск не сообщит, что все данные и метаданные оказались на диске.
	\end{itemize}
\end{frame}

\begin{frame}{Коды ошибок fsync}
	В случае успеха возвращается \textbf{0}, в противном возвращается \textbf{–1}.
	
	\medskip	
	Устанавливается значение errno:
	\begin{itemize}
		\item \textbf{EBADF} -- указанный дескриптор файла не является допустимым дескриптором, открытым для записи;
		\item \textbf{EINVAL} -- указанный дескриптор файла отображается на объект, не поддерживающий синхронизацию;
		\item \textbf{EIO} -- при синхронизации произошла низкоуровневая ошибка ввода­-вывода.
	\end{itemize}
\end{frame}

\begin{frame}[fragile]{Синхронный ввод-вывод}
	\begin{block}{sync()}
		обеспечивает синхронизацию всех буферов, имеющихся на диске
	\end{block}
	
	\begin{minted}{c}
	#include <unistd.h>

	void sync(void);
	\end{minted}
	
	\begin{itemize}
		\item функция не имеет ни параметров, ни возвращаемого значения. 
		\item функция всегда завершается успешно, и после ее возврата все буферы -- содержащие как данные, так и метаданные -- гарантированно оказываются на диске.
	\end{itemize}
\end{frame}

\begin{frame}[fragile]{Флаг синхронизации}
	Флаг O\_SYNC может быть передан вызову \textbf{open}(). 
	
	Этот флаг означает, что все операции ввода-­вывода, осуществляемые с этим файлом, должны быть синхронизированы.

	\begin{minted}{c}
	int fd;

	fd = open(file, O_WRONLY | O_SYNC);	
	if (fd == -1) {
	  perror("open");
	  return -1;
	}
	\end{minted}

	\begin{itemize}
		\item запросы на считывание синхронизированы \textbf{всегда}. 
		\item вызовы write(), как правило, не синхронизируются.
	\end{itemize}
	
	Флаг O\_SYNC можно рассмотреть в следующем ключе: он принудительно выполняет неявный вызов \textbf{fsync}() после каждой операции \textbf{write}() перед возвратом вызова. 
\end{frame}

\section{Закрытие файлов}
\begin{frame}
  \frametitle{Содержание лекции}
  \tableofcontents[current]
\end{frame}

\begin{frame}[fragile]
	Системный вызов \textbf{close()} -- разорвать связь между дескриптором и файлом, который с ним ассоциирован.
	\begin{minted}{c}
	#include <unistd.h>

	int close(fd);
	\end{minted}
	\begin{itemize}
		\item \textbf{close}() отменяет отображение открытого файлового дескриптора \textbf{fd} и разрывает связь между файлом и процессом. 
		\item ядро свободно может переиспользовать дескриптор как возвращаемое значение для последующих вызовов \textbf{open}() или \textbf{creat}(). 
		\item успешное выполнение -- возвращается \textbf{0}, иначе \textbf{-1}. 
	\end{itemize}
	\begin{minted}{c}
	if (close(fd) == -1)
	  perror("close");
	\end{minted}
	\begin{itemize}
		\item Закрытие файла никак не связано с актом сбрасывания файла на диск. 
	\end{itemize}
\end{frame}

\section{Позиционирование, усечение}
\begin{frame}
  \frametitle{Содержание лекции}
  \tableofcontents[current]
\end{frame}

\begin{frame}[fragile]
	Системный вызов \textbf{lseek}()  -- установка файловой позиции файлового дескриптора. 

	\begin{minted}{c}
	#include <sys/types.h>
	#include <unistd.h>
	off_t lseek(int fd, off_t pos, int origin);
	\end{minted}

	Аргумент \textbf{origin}:
	\begin{itemize}
		\item SEEK\_CUR - текущая файловая позиция дескриптора \textbf{fd} установлена в его текущее значение плюс \textbf{pos}. 
		
		Если \textbf{pos} = 0, то возвращается текущее значение файловой позиции.
		\item SEEK\_END - текущая файловая позиция дескриптора \textbf{fd} установлена в текущее значение длины файла плюс \textbf{pos}.
		
		Если \textbf{pos} = 0, то смещение устанавливается в конец файла.
		\item SEEK\_SET - текущая файловая позиция дескриптора fd установлена в pos. 
		
		Если \textbf{pos = 0}, то смещение устанавливается в начало файла. 		
	\end{itemize}
\end{frame}

\begin{frame}[fragile]
	Файловая позиция дескриптора \textbf{fd} устанавливается равной \textbf{1825}:

	\begin{minted}{c}
	off_t ret;	
	ret = lseek(fd, (off_t) 1825, SEEK_SET);	
	if (ret == (off_t) -1)
	  /* ошибка */ 
	\end{minted}
	
	Установить файловую позицию дескриптора fd в конец файла:
	
	\begin{minted}{c}
	off_t ret;	
	ret = lseek(fd, 0, SEEK_END);	
	if (ret == (off_t) -1)
	  /* ошибка */ 
	\end{minted}
	
	Получить текущую файловую позицию:

	\begin{minted}{c}
	off_t pos;	
	pos = lseek(fd, 0, SEEK_CUR);	
	if (pos == (off_t) -1)
	  /* ошибка */ 
	else
	  /* pos - текущая позиция fd */
	\end{minted}
\end{frame}

\end{document}
